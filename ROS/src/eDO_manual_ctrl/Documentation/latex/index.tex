This R\+OS package can be used to control the C\+O\+M\+AU e.\+DO educational robot. It provides the ability to initialize, calibrate, and operate the e.\+DO from the Linux terminal without the use of the Android tablet application. The program supports jog and move commands and can output data from the e.\+DO to the terminal. It can be helpful in understanding how the e.\+DO Robot can be controlled by any R\+OS compatible program.

\subsection*{Getting Started}

These instructions will get you a copy of the project up and running on your local machine for development and testing purposes.

\subsubsection*{Prerequisites}

This package has only been run on the N\+V\+I\+D\+IA Jetson T\+X2 running Ubuntu 16.\+04 L\+TS and R\+OS Kinetic. In order to use the jog mode, the Ncurses library must be installed on your Linux machine.

e.\+DO Software Version\+: 2.\+1.\+0

\subsubsection*{Installing}

Clone/save this repository into the \char`\"{}src\char`\"{} folder in a catkin workspace directory. Use catkin\+\_\+make from your catkin workspace directory to build.

Clone/save directory into src folder


\begin{DoxyCode}
1 cd catkin\_ws/src/
2 git clone https://github.com/jshelata/eDO\_manual\_ctrl.git
\end{DoxyCode}


Build the package


\begin{DoxyCode}
1 cd catkin\_ws
2 catkin\_make
\end{DoxyCode}


Connect to your e.\+DO\textquotesingle{}s Wi\+Fi network and set the R\+O\+S\+\_\+\+M\+A\+S\+T\+E\+R\+\_\+\+U\+RI of your machine to the e.\+DO\textquotesingle{}s IP address and the R\+O\+S\+\_\+\+IP to your machine\textquotesingle{}s IP on the e.\+DO Wi\+Fi network. N\+O\+TE\+: You can also add these two lines to the bottom of your .bashrc in your home directory.


\begin{DoxyCode}
1 export ROS\_MASTER\_URI=http://192.168.12.1:11311
2 export ROS\_IP=192.168.12.68
\end{DoxyCode}


Source your setup.\+bash file within your catkin workspace.


\begin{DoxyCode}
1 source devel/setup.bash
\end{DoxyCode}


Run the node.


\begin{DoxyCode}
1 rosrun edo\_manual\_ctrl edo\_manual\_ctrl
\end{DoxyCode}
 \paragraph*{L\+AN Connection}

If you prefer to access the e.\+DO over your own L\+AN using an Ethernet cable, you mush change the R\+O\+S\+\_\+\+M\+A\+S\+T\+E\+R\+\_\+\+U\+RI and R\+O\+S\+\_\+\+IP within the e.\+DO as well as on your own machine.

Connect to e.\+DO via ssh using e.\+DO\textquotesingle{}s IP on your L\+AN. N\+O\+TE\+: It may be helpful to reserve an IP on your router for the e.\+DO. 
\begin{DoxyCode}
1 ssh edo@10.42.0.49
\end{DoxyCode}
 The e.\+DO\textquotesingle{}s password is \char`\"{}raspberry\char`\"{}.

Next, you\textquotesingle{}ll need to edit the \char`\"{}ministarter\char`\"{} file in the home directory. Change the IP address in lines 20 and 21 to e.\+DO\textquotesingle{}s IP address on the L\+AN. Save and close the file and restart e.\+DO. Change the R\+O\+S\+\_\+\+M\+A\+S\+T\+E\+R\+\_\+\+U\+RI as explained above to the new IP address and the R\+O\+S\+\_\+\+IP to you machine\textquotesingle{}s IP address on the L\+AN.

\subsection*{Build Dependencies}


\begin{DoxyItemize}
\item \href{https://www.cyberciti.biz/faq/linux-install-ncurses-library-headers-on-debian-ubuntu-centos-fedora/}{\tt Ncurses} -\/ Used for asynchronous jog control
\item \href{http://wiki.ros.org/kinetic.Installation}{\tt R\+OS Kinetic}
\item \href{https://github.com/Comau/eDO_core_msgs}{\tt e\+D\+O\+\_\+core\+\_\+msgs} -\/ Must be cloned in catkin\+\_\+ws/src/
\item C++11 -\/ Used to sleep to give time for e.\+DO to process commands
\end{DoxyItemize}

\subsection*{Authors}


\begin{DoxyItemize}
\item {\bfseries Jack Shelata} -\/ \href{mailto:jack@jackshelata.com}{\tt jack@jackshelata.\+com}
\end{DoxyItemize}

\subsection*{License}

This project is licensed under the M\+IT License -\/ see \hyperlink{md_LICENSE}{L\+I\+C\+E\+N\+SE.md} file for details 